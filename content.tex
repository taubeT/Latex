%*******************************************************
\chapter{Einleitung}
%*******************************************************
\section{Die Motivation}
Als \textit{Exergames} bezeichnet man Videospiele, die nur durch ein aktives Mitwirken des Spielers in Form von realen Aktivit�tshandlungen spielbar sind \cite{Considerations}. Die Idee von Mobile Exergames liegt darin, physikalische Bewegungen mit Videospielen auf mobilen Ger�ten, wie z.B. Smartphones, zu verkn�pfen \cite{MobileExergames}. Mobile Exergames k�nnen deren Benutzer auf spielerischem Wege dazu veranlassen, sich mehr zu bewegen \cite{MobileExergaming}, denn wissenschaftlichen Statistiken zufolge \cite{SittingIsSmoking} bringen mangelnde Bewegung hohe Gesundheitsrisiken mit sich. Doch auch Mobile Exergames k�nnen diesen Zweck f�r die Gesundheit des Menschen nicht erf�llen, wenn niemand die Spiele spielt. Daher bleibt die entscheidende Frage,wie Menschen davon �berzeugt werden k�nnen, Mobile Exergames zu spielen. \\

Dieser Frage widmet sich das \textit{Behavior Model (FBM)} vom Experimentalpsychologen BJ Fogg \cite{BehaviorModel} und zeigt zugleich auf, welche Faktoren notwendig sind, um menschliches Verhalten anzuregen. Nach dem Modell \cite{BehaviorModel} sind es die drei Faktoren \textit{Motivation}, \textit{F�higkeit} und \textit{Trigger}, die einen Menschen dazu veranlassen, ein bestimmtes Verhalten zu zeigen. Die Motivation ist die "Triebkraft" und das Streben des Menschen nach w�nschenswerten Zielen \cite{Motivation}. Die F�higkeit bezeichnet das Verm�gen einer Person, etwas zu tun. Das Triggering ist ein wesentlicher Bestandteil des FBM Model und besch�ftigt sich mit zwei wichtigen Fragen: Wann soll die Person informiert werden und wie soll die Person informiert werden. Nur wenn alle Faktoren aufeinander abgestimmt sind und miteinander harmonieren, wird sich, nach dem FBM Model \cite{BehaviorModel}, das zu erwartende Verhalten beim Menschen zeigen. \\

Ausf�hrliche Forschungen �ber den Zusammenhang und die Integration des Triggering Faktors mit Mobile Exergames sind notwendig, damit Menschen dazu veranlasst werden, diese Spiele zu spielen. Auf der einen Seite sind Entwicklungen auf diesem Gebiet der Forschung von besonderer Bedeutung f�r erfolgreiche Mobile Exergames, auf der anderen Seite werden die Ergebnisse der Forschung zugleich auch andere Gebiete (z.B. kontext-aware mobile Applikationen) beeinflussen. \\


 \section{Die Zielsetzung}
 Ziele der Masterarbeit sind: 
 \begin{itemize}
 \item Ausf�hrliche Forschungen des vom Experimentalpsychologen BJ Fogg entwickelten Behavior Model (FBM).
 \item Ausf�rliche Forschungen �ber die auf Smartphones beruhenden kontext-aware Systeme der Mobile Exergames.
 \item Entwicklung eines Triggering Konzepts sowie eines Algorithmus f�r kontext-aware Mobile Exergames.
 \item Testen des entwickelten Triggering Systems und des Algorithmus.
 \item Er�rterung und Evaluierung der Ergebnisse. 
 \end{itemize}


 \section{Die Gliederung der Arbeit}
 //TODO Gliederung der Masterarbeit
%*******************************************************
\chapter{Stand der Forschung}
%*******************************************************
\section{Kontext}
\#Kontext hat vier wesentliche Kategorien, Identit�t, Lokation, Status (oder Activit�t) und die Zeit \cite{Motivation}. 
\subsection{Welche Variablen sollen benutzt werden}
//TODO FinaAllVariabler1 => Kalendar, Zeit, Mikrofon, GPS, Schritterkennung\newline
//TODO FindAllVariabler2 => Hareware : microphone, GSM Radio, Key pressed
                            Software : Kalender, incoming SMS 

\section{Design}
\section{Umsetzung}
\section{Evaluation}
\section{Zusammenfassung}
